\documentclass [11pt]{book}
\usepackage[utf8]{inputenc}

\usepackage[magyar]{babel}
\usepackage{t1enc}
\frenchspacing

\usepackage{hulipsum}
\usepackage{fancyhdr}
\fancyhead[L]{}
\fancyhead[RE]{\nouppercase{\leftmark}}
\fancyhead[RO]{\nouppercase{\rightmark}}
\fancyfoot[R]{\thepage}

\usepackage{graphicx}
\usepackage{subcaption}
\usepackage{wrapfig}

\usepackage{array}
\usepackage[table]{xcolor}
\usepackage{multirow}
\usepackage{xcolor}
\usepackage{amsmath}
\usepackage{mathtools}
\usepackage{amsfonts}
\usepackage{amsthm}
\theoremstyle{definition}
\newtheorem{tet}{Tétel}
\newtheorem{defin}{Definíció}
\usepackage{chngcntr}
\counterwithout{equation}{chapter}

\usepackage{listings}

\usepackage{algpseudocode}

\title{Zárthelyi dolgozat\\\Large A csoport}
\author{Bordás Milán WJB0DC}
\date{\today}

\begin{document}
\pagestyle{fancy}
\maketitle

\chapter{1. Feladat}

\section{1. Szekció}
\hulipsum[1-5]
\section{2. Szekció}
\hulipsum[1-5]

\chapter{2. Feladat}
\hulipsum[1-2]
\begin{wrapfigure}[25]{L}[0cm]{10cm}
\centering
\caption{Ez itt az ábra}
\begin{subfigure}{5cm}
\centering
\caption{Ez a részábra}
\includegraphics[height=4cm, scale=0.5]{szines.jpg}
\end{subfigure}
\\
\begin{subfigure}{5cm}
\centering
\caption{Ez is részábra}
\fbox{\includegraphics[height=4cm]{szepia.jpg}}
\end{subfigure}
\end{wrapfigure}
\hulipsum[1-2]
\chapter{3. feladat}
\begin{defin}[Sajátérték]

Legyen $A,\in \mathbb{R}^{n \times n}$, négyzetes mátrix. Azt mondjuk,hogy $\lambda \in \mathbb{C} $ sajátértéke és $v \in \mathbb{C}^n$
a $\lambda$ sajátértékhez tartozó (jobb oldali) sajátvektora $A$-nak, ha
\begin{equation} 
Av = \lambda v
\end{equation}
\end{defin}

\begin{defin}[Karakterisztikus polinom]
Jelölje $E \in \mathbb{R}^{n \times n}$ az egységmátrixot. Az $A$ ún. \textit{karakterisztikus polinomja}
\begin{equation} 
\newcommand{\rlambda}{\textcolor{red}{-\lambda}}
\varphi (\lambda) := \det(A\textcolor{red}{-\lambda E}) = 
\begin{array}{|cccc|}
a_{11} \rlambda & a_{12} & \dots & a_{1n} \\
a_{21} & a_{22} \rlambda & \dots & a_{2n} \\
\vdots & \vdots & \ddots & \vdots \\
a_{21} & a_{22}  & \dots & a_{nn} \rlambda \\
\end{array}
,
\end{equation}
egy $n$-edfokú polinom $\lambda$-ban.
\end{defin}

\begin{tet}[Sajátértékek meghatározása]
Az $A \in \mathbb{R}^{n \times n}$ mátrix sajátértékei az ún. \textit{karakterisztikus egyenlet}
\begin{equation}
\varphi (\lambda) = 0
\end{equation}
megoldásai. Mivel a $\varphi (\lambda)$ \textit{karakterisztikus polinom} egy $n$-edfokú polinom $\lambda$-ban, ezért a komplex számokon (multiplicitással együtt) $n$ megoldása van.
\end{tet}

\chapter{4. feladat}
\hulipsum[1-3]
\begin{lstlisting}[language=c, numbers=right, stepnumber=2, tabsize=2, showspaces=true, frame=shadowbox,
title={Programkód 1. Bináris keresés C-ben}, float=hbt!]
binarySearch(arr, x, low, high)
	repeat till low = high
		mid = (low + high)/2
			if (x == arr[mid])
				return mid

			else if (x > arr[mid])	// x is on the right side
				low = mid + 1

			else			// x is on the left side
				high = mid - 1
\end{lstlisting}
\hulipsum[1-2]
\chapter{5. feladat}
\newenvironment{vers}[2]
{
  \centering {\textbf{#1}\\\textit{#2}}
  
}
{}

\begin{vers}{Bordás Milán}{Írom a ZH-t}
\hulipsum[1]
\end{vers}

\chapter{Bónusz feladat}
\newcounter{index}
\setcounter{index}{1}
\whiledo{\value{index}<61}
{
\ifthenelse{\isodd{\value{index}}}{\textcolor{black}{\theindex}}{\textcolor{red}{\theindex}}
\stepcounter{index}
}
\end{document}
